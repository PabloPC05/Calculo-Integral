\documentclass[a4paper, 12pt]{article}
\usepackage{amsmath, amssymb, amsfonts, amsthm}
\usepackage{graphicx}
\usepackage{hyperref}
\usepackage{multicol}
\usepackage{soul}

% Definir estilos de teoremas con amsthm
\theoremstyle{definition}
\newtheorem{definition}{Definición}[section]
\newtheorem{remark}{Observación}[section]

\theoremstyle{plain} % Para teoremas en negrita y cursiva
\newtheorem{theorem}{Teorema}[section]

\title{Cálculo Integral}
\author{Pablo Pardo Cotos}

\begin{document}

\maketitle
\tableofcontents
\newpage

\section{Tema 1: Teoría de la Integral}

\begin{definition}[Rectángulo n-dimensional]
    Un \textbf{rectángulo n-dimensional} en el espacio euclídeo \(\mathbb{R}^n\) es un conjunto de la forma:
    \[
    R = [a_1, b_1] \times [a_2, b_2] \times \dots \times [a_n, b_n] = 
    \{ (x_1, x_2, \dots, x_n) \in \mathbb{R}^n : a_i \leq x_i \leq b_i, i = 1, 2, \dots, n \}.
    \]
\end{definition}

\begin{definition}[Volumen de un rectángulo]
El \textbf{volumen} de un rectángulo \( R \) en \(\mathbb{R}^n\) se define como:
\[
V(R) = \prod_{i=1}^{n} (b_i - a_i).
\]
\end{definition}

\begin{definition}[Rectángulo degenerado]
Si el volumen de \( R \) es \( 0 \), decimos que el rectángulo es \textbf{degenerado}.
\end{definition}

\begin{remark}
El volumen de \( R_\delta \) es:
\[
V(R_\delta) = \prod_{i = 1}^{n} (b_i + \delta - a_i + \delta) = \prod_{i = 1}^{n} (b_i - a_i + 2\delta) = V(R) + 2n\delta.
\]
Tomando el límite cuando \( \delta \to 0 \), obtenemos:
\[
\lim_{\delta \to 0} V(R_\delta) = V(R).
\]
\end{remark}

\begin{definition}[Medida exterior de Lebesgue]
Sea \( A \subset \mathbb{R}^n \) un conjunto cualquiera, definimos la \textbf{medida exterior de Lebesgue} como:
\[
m^*(A) = \inf \left\{ \sum_{i = 1}^{\infty} V(R_i) : A \subset \bigcup_{i = 1}^{\infty} R_i \right\}.
\]
\end{definition}

\begin{remak}
    \begin{enumerate}
        \item Si $m^{*}(A) = +infty \iff \forall (R_j)_{j \in \mathbb{N}}$ sucesión de rectángulos n-dimensionales con $A \subset \bigcup_{j \in \mathbb{N}} R_j$ se tiene que $\sum_{j = 1}^{\infty} V(R_j) = +\infty$.
\end{remark}

\end{document}
